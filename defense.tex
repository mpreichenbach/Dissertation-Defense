\documentclass{beamer}
\usetheme{Madrid}

\usepackage[utf8]{inputenc}
\usepackage{xcolor}
\usepackage{graphicx}
\usepackage{biblatex}
\addbibresource{references.bib}

\title[Dissertation Defense]{Spectral Properties of a Non-compact Operator in Ecology}
\author[Matt Reichenbach]{Matt Reichenbach \\ \, \\ Advised by Richard Rebarber and Brigitte Tenhumberg}
\date[November 25, 2020]{Dissertation Defense, November 25, 2020}

\begin{document}

\frame{\titlepage}

%\begin{frame}
%	\frametitle{}
%	\begin{itemize}
%		\item 
%	\end{itemize}
%\end{frame}

\begin{frame}
\frametitle{Introduction}
	\begin{itemize}
		\item Integral projection models (IPMs) are stage-structured population models of the form
		\[\varphi_{t+1} = A \varphi_t := \int_L^U k(y, x) \varphi_t(x) \, dx,\]
		\pause
		where $\varphi_t$ gives the population distribution at time $t$, the limits $L$, $U$ are the lower- and upper-limits of the structure variable $x$, and the kernel $k(y, x)$ determines how individuals of size $x$ contribute to those of size $y$ in the next time step.
		\pause
		\item IPMs generalize Leslie matrices by allowing for a continuous structure variable.
	\end{itemize}
\end{frame}

\begin{frame}
	\frametitle{Introduction}
	\begin{itemize}
		\item We will consider kernel functions of the form
		\[k(y, x) = s(x) g(y, x) + b(y)f(x),\]
		\pause
		where 
		\pause
		\begin{itemize}
			\item $s(x)$ is the survival function,
			\pause
			\item $g(y, x)$ is the growth subkernel,
			\pause
			\item $b(y)$ is the offspring distribution, 
			\pause
			\item and $f(x)$ the fecundity function.	
		\end{itemize}
	\end{itemize}
\end{frame}

\begin{frame}
	\frametitle{Introduction}
	\begin{itemize}
		\item When individuals are allowed to increase or decrease over the time step, the kernel function $k(y, x)$ of the IPM operator $A$ will be bounded.
		\pause
		\item For example, this is the case when the structure variable $x$ is stem diameter, or individual biomass.
		\pause
	\end{itemize}
\end{frame}

\begin{frame}
	\frametitle{Introduction}
		In this case, Ellner \& Rees, in appendices to \cite{Ellner2006}, proved the theorem:
	\pause
	\begin{theorem}
		Suppose $A$ is an IPM operator whose kernel $k(y, x)$ is positive and continuous on $[L, U]^2$. Then $\lambda = r(A)$ is an eigenvalue of $A$, and its eigenvector $\psi$ can be scaled to be positive. Additionally, $\lambda$ is the asymptotic growth rate of the population, and $\psi$ is the stable stage distribution, in the sense that for any nonzero initial population $\varphi_0$,
		\[\lim_{n \to \infty} \frac{A^n \varphi_0}{\lambda^n} = \langle \psi, \varphi_0 \rangle \psi. \]
	\end{theorem}
\end{frame}

\begin{frame}
	\frametitle{Introduction}
	\begin{itemize}
		\item However, ecologists have since used IPMs to model individuals that \emph{cannot} shrink from one time step to the next.
		\pause
		\item For example, Vindenes et. al in \cite{Vindenes2014} modeled fish, and used length as the structure variable $x$.
		\pause
		\item Fish have bony skeletons, and hence cannot shrink in length.
	\end{itemize}
\end{frame}

\begin{frame}
	\frametitle{Introduction}
	\begin{itemize}
		\item Ellner \& Rees's theorem implicitly assumed that individuals could shrink; if they cannot shrink the kernel will be unbounded.
		\pause
		\item All IPMs assume that $g(\cdot, x)$ is a probability distribution; that is:
		\[\int_L^U g(y, x) \, dy = 1, \quad \text{for all } x \in [L,U].\]
		\pause
		\item The assumption that individuals cannot shrink is that
		\[g(y, x) = 0, \text{whenever } y < x.\]
	\end{itemize}
\end{frame}

\begin{frame}
	\frametitle{Introduction}
\begin{center}
	\includegraphics[width=\linewidth]{"Images/Growth Kernel Plot"}
\end{center}
\end{frame}

\begin{frame}
	\frametitle{Introduction}
	\begin{itemize}
		\item With an IPM which models a population whose individuals cannot shrink, we have some questions to answer:
		\pause
		\begin{itemize}
			\item Is the operator $T$ still compact?
			\pause
			\item Is $\lambda = r(T)$ still an eigenvalue of $T$?
			\pause
			\item Are $\lambda$ and its eigenvector $\psi$ still the asymptotic growth rate and stable stage distribution, respectively, of the population?
		\end{itemize}
	\end{itemize}
\end{frame}

\begin{frame}
	\frametitle{$T$ is Not Compact}
	\begin{itemize}
		\item 
	\end{itemize}
\end{frame}

\begin{frame}[allowframebreaks]
	\frametitle{References}
		\printbibliography
\end{frame}

%\begin{frame}
%	\frametitle{}
%	\begin{itemize}
%		\item 
%	\end{itemize}
%\end{frame}


\end{document}









