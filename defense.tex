\documentclass{beamer}
\usetheme{Madrid}

\usepackage[utf8]{inputenc}
\usepackage{xcolor}
\usepackage{graphicx}

\title[Density Dependent Growth]{A Limit-set Trichotomy for a Density-Dependent Integral Projection Model}
\author[Matt Reichenbach]{Matt Reichenbach}
\date[\today]{STAARS, \today}

\begin{document}

\frame{\titlepage}

%\begin{frame}
%	\frametitle{}
%	\begin{itemize}
%		\item 
%	\end{itemize}
%\end{frame}

\begin{frame}
\frametitle{Introduction}
	\begin{itemize}
		\pause
		\item Way back in Fall 2017, Richard mentioned that he had proved some results for a nonlinear matrix model.
		\pause
		\item He suggested I try to prove similar results for an integral projection model.
		\pause
		\item Before I could prove those results, we ran into another problem which has become the bulk of my research here.
	\end{itemize}
\end{frame}

\begin{frame}
	\frametitle{Matrix Model}
	\begin{itemize}
		\pause
		\item Richard and some colleagues wanted to study ``stunting" in fish, where individuals show inhibited growth when total biomass is large..
		\pause
		\item They simplified pre-existing models to make it mathematically tractable; they ended up studying a system
		\[n_{t+1} = A_{p_t} n_t, \quad \text{where } p_t = g(B_t),\]
		\pause
		where $B_t$ is the biomass of the population vector $n_t$, and $g(\cdot)$ is a differentiable, strictly decreasing function with $g(0)=1$ and $\lim_{y \to \infty} g(y) = 0$.
		\pause
		\item The value $p_t$ is the probability that an individual grows to the next size class in a time step.
	\end{itemize}
\end{frame}

\begin{frame}
	\frametitle{Matrix Model}
	\begin{itemize}
		\pause
		\item Here's a $4 \times 4$ version of the model they studied:
		\[n_{t + 1} = \begin{pmatrix} 0 & f_1 & f_2 & f_3 \\ s_0 & s_1(1 - p_t) & 0 & 0 \\ 0 & s_1 p_t & s_2 (1 - p_t) & 0 \\ 0 & 0 & s_2 p_t & s_3 \end{pmatrix} n_t,\]
		\pause
		with $p_t = g(B_t)$.
	\end{itemize}
\end{frame}

\begin{frame}
	\frametitle{Matrix Model}
	\begin{itemize}
		\pause 
		\item Here is the theorem they proved:
		\pause
		\begin{theorem}
			For the system given above, the following hold:
			\begin{enumerate}
				\pause
				\item for $0 \leq p \leq q \leq 1$, $r(A_p) \leq r(A_q)$;
				\pause
				\item if $r(A_1) < 1$, then the zero population $n_0 \equiv 0$ is globally asymptotically stable;
				\pause
				\item if $r(A_0) > 1$, then $\lim_{t \to \infty} ||n_t|| = \infty$ for all nonzero nonnegative initial states $n_0$;
				\pause
				\item if $r(A_0) < 1 < r(A_1)$, then the system is bounded and has a unique nonzero equilibrium $n^*$, and for every nonnegative initial population $n_0$, the biomass converges to the equilibrium biomass $B^*$, where $B^*$ is the biomass of the equilibrium population $n^*$.
			\end{enumerate}
		\end{theorem}
	\end{itemize}
\end{frame}

\begin{frame}
	\frametitle{Matrix Model}
	\begin{itemize}
		\pause
		\item The properties (2)-(4) are an example of a ``limit-set trichotomy", meaning it gives conditions for when the system goes to zero, a nonzero equilibrium, or infinity.
		\pause
		\item Richard then suggested I investigate whether something similar held for integral projection models.
	\end{itemize}
\end{frame}

\begin{frame}
	\frametitle{Integral Projection Model}
	\begin{itemize}
		\item IPM's have the form
		\[n_{t + 1}(y) = \int_L^U k(y, x) n_t(x) \, dx,\]
		\pause
		where
		\[k(y, x) = s(x) g(y, x) + d(y) f(x).\]

	\end{itemize}
\end{frame}

\begin{frame}
	\frametitle{Integral Projection Model}
	\begin{itemize}
		\pause
		\item Since I wanted to model density-dependent growth, I focused on the growth function $g(y, x)$.
		\pause
		\item To start, I used an example from the literature for the linear model:
		\[	g(y, x) = \frac{1}{\sqrt{2 \pi}(y - x) v(x)} \exp  \left( - \frac{(\ln(y - x) - \mu(x))^2}{2 v(x)} \right).\]
	\end{itemize}
\end{frame}

\begin{frame}
	\frametitle{Integral Projection Model}
	\begin{itemize}
		\pause
		\item I then altered it so that the average growth increment $m(x) - x$ would be scaled by $p_t = g(B_t)$:
		\pause
		\begin{align*}
			g(y, x, p_t) &:= \frac{1}{\sqrt{2 \pi}(y - x) v_{p_t}(x)} \exp \left( - \frac{(\ln(y - x) - \mu_{p_t}(x))^2}{2 v_{p_t}(x)} \right), \\
			\mu_{p_t}(x) &:= \log \left( \frac{((p_t(m(x)-x))^2}{\sqrt{((p_t(m(x) - x))^2 + \sigma(x)^2}}\right),\\
			v_{p_t}(x) &:= \log \left( 1 + \frac{\sigma(x)^2}{(p_t(m(x)-x))^2}\right).
		\end{align*}
		\pause where $m(x)$ is the average expected size for an individual of size $x$ to grow to, and $\sigma(x)$ is the standard deviation of the sizes on a linear scale.
	\end{itemize}
\end{frame}

\begin{frame}
	\frametitle{Integral Projection Model}
	\begin{itemize}
		\item This looks insane, so here's a plot:
\begin{figure}
	\centering
	\includegraphics[width=0.8\linewidth]{"Growth Kernel Plot"}
\end{figure}
	\end{itemize}
\end{frame}

\begin{frame}
	\frametitle{Integral Projection Model}
	\begin{itemize}
		\pause
		\item It turns out that for any fixed $p_t$, the linear operator $T_{p_t}$ isn't compact, so we didn't know whether the spectral radius $r(T_{p_t})$ was even an eigenvalue (which it is in the matrix model).
		\pause
		\item I spent most of my time over the past few years showing that the non-compact operator does have its spectral radius as an eigenvalue.
		\pause
		\item But you've probably seen that talk before.
	\end{itemize}
\end{frame}

\begin{frame}
	\frametitle{Simulation Results: $r(T_1) < 1$}
	\begin{itemize}
		\item First, I wanted an IPM operator $T$ such that $r(T_1) < 1$; I achieved this by making the fecundity $0.05f(x)$.
		\pause
		\item Comparing this with the matrix model, I expected the population to die out.
	\end{itemize}
\end{frame}

\begin{frame}
	\frametitle{Simulation Results: $r(T_1) < 1$)}
	\begin{itemize}
			\pause
			\item The spectral radii (growth rates) stay below 1:
			\pause
			\begin{figure}
				\centering
				\includegraphics[width=0.7\linewidth]{F=0.05/spectral_radius_when_f=0.05}
			\end{figure}
	\end{itemize}
\end{frame}

\begin{frame}
	\frametitle{Simulation Results: $r(T_1) < 1$}
	\begin{itemize}
		\pause
		\item Also the population and biomass go to zero:
		\pause
		\begin{figure}[H]
			\centering
			\begin{minipage}{.4\textwidth}
				\includegraphics[width=\linewidth]{F=0.05/total_pop_when_f=0.05}
				\caption{Total population}
			\end{minipage} \quad 
			\centering
			\begin{minipage}{.4\textwidth}
				\includegraphics[width=\linewidth]{F=0.05/total_biomass_when_f=0.05}
				\caption{Total biomass}
			\end{minipage}
		\end{figure}
	\end{itemize}
\end{frame}

\begin{frame}
	\frametitle{Simulation Results: $r(T_1) < 1$}
	\begin{itemize}
		\pause
		\item And the stable stage distribution (the leading eigenvector of $T_1$) shows no evidence of stunting:
		
		\begin{figure}
			\centering
			\includegraphics[width=0.7\linewidth]{F=0.05/ssd_when_f=0.05}
		\end{figure}
	\end{itemize}
\end{frame}

\begin{frame}
	\frametitle{Simulation Results: $r(T_{0.001}) < 1 < r(T_1)$}
	\begin{itemize}
		\pause 
		\item Next, I investigated whether there would be an equilibrium population when $r(T_{0.001}) < 1 < r(T_1)$.
		\pause
		\item This happened when I left the fecundity unchanged.
	\end{itemize}
\end{frame}

\begin{frame}
	\frametitle{Simulation Results: $r(T_{0.001}) < 1 < r(T_1)$}
	\begin{itemize}
		\pause 
		\item In this case, the growth rates approach 1:
		
\begin{figure}
	\centering
	\includegraphics[width=0.7\linewidth]{F=1/spectral_radius_when_f=1}
\end{figure}
	\end{itemize}
\end{frame}

\begin{frame}
	\frametitle{Simulation Results: $r(T_{0.001}) < 1 < r(T_1)$}
	\begin{itemize}
		\pause 
		\item The population and biomass do approach positive equilibrium values:
		\begin{figure}[H]
			\centering
			\begin{minipage}{.4\textwidth}
				\includegraphics[width=\linewidth]{F=1/total_pop_when_f=1}
				\caption{Total population}
			\end{minipage} \quad 
			\centering
			\begin{minipage}{.4\textwidth}
				\includegraphics[width=\linewidth]{F=1/total_biomass_when_f=1}
				\caption{Total biomass}
			\end{minipage}
		\end{figure}
	\end{itemize}
	\end{frame}

\begin{frame}
	\frametitle{Simulation Results: $r(T_{0.001}) < 1 < r(T_1)$}
	\begin{itemize}
		\pause 
		\item In this case, the equilibrum distribution shows evidence of stunting:
\begin{figure}
	\centering
	\includegraphics[width=0.7\linewidth]{F=1/ssd_when_f=1}
\end{figure}

	\end{itemize}
\end{frame}

\begin{frame}
	\frametitle{Simulation Results: $1 < r(T_{0.001})$}
	\begin{itemize}
		\pause 
		\item In this case, I investigated whether the population would go to infinity.
		\pause
		\item I was able to guarantee $1 < r(T_{0.001})$ when I made the fecundity $1000f(x)$.
	\end{itemize}
\end{frame}

\begin{frame}
	\frametitle{Simulation Results: $1 < r(T_{0.001})$}
	\begin{itemize}
		\pause
		\item Here, the growth rates stayed above 1 (eventually reaching $r(T_{0.001}))$:
		
\begin{figure}
	\centering
	\includegraphics[width=0.7\linewidth]{F=1000/spectral_radius_when_f=1000}
\end{figure}
	\end{itemize}
\end{frame}

\begin{frame}
	\frametitle{Simulation Results: $1 < r(T_{0.001})$}
	\begin{itemize}
		\pause
		\item The total population and biomass continue to increase without bound:
		
\begin{figure}[H]
	\centering
	\begin{minipage}{.4\textwidth}
		\includegraphics[width=\linewidth]{F=1000/total_biomass_when_f=1000}
		\caption{Total population}
	\end{minipage} \quad 
	\centering
	\begin{minipage}{.4\textwidth}
		\includegraphics[width=\linewidth]{F=1000/total_biomass_when_f=1000}
		\caption{Total biomass}
	\end{minipage}
\end{figure}
	\end{itemize}
\end{frame}

\begin{frame}
	\frametitle{Simulation Results: $1 < r(T_{0.001})$}
	\begin{itemize}
		\pause
		\item The steady state distribution shows even more significant stunting:
		\pause
		
\begin{figure}
	\centering
	\includegraphics[width=0.7\linewidth]{F=1000/ssd_when_f=1000}
\end{figure}
		
	\end{itemize}
\end{frame}


\begin{frame}
	\frametitle{THE END}
	\begin{center}
Any questions or comments?
\end{center}
\end{frame}

\end{document}









