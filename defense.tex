\documentclass{beamer}
\usetheme{Madrid}

\usepackage[utf8]{inputenc}
\usepackage{amsmath, mathrsfs}
\usepackage{xcolor}
\usepackage{graphicx}
\usepackage{biblatex}
\addbibresource{references.bib}

\newcommand{\R}{\mathbb{R}}
\newcommand{\C}{\mathbb{C}}
\newcommand{\N}{\mathbb{N}}
\newcommand{\dd}{{\delta}}
\newcommand{\ee}{{\varepsilon}}

\title[Dissertation Defense]{Spectral Properties of a Non-compact Operator in Ecology}
\author[Matt Reichenbach]{Matt Reichenbach \\ \, \\ Advised by Richard Rebarber and Brigitte Tenhumberg}
\date[November 25, 2020]{Dissertation Defense, November 25, 2020}

\begin{document}

\frame{\titlepage}

%\begin{frame}
%	\frametitle{}
%	\begin{itemize}
%		\item 
%	\end{itemize}
%\end{frame}

\begin{frame}
\frametitle{Introduction}
	\begin{itemize}
		\item Integral projection models (IPMs) are stage-structured population models of the form
		\[\varphi_{t+1} = A \varphi_t := \int_L^U k(y, x) \varphi_t(x) \, dx,\]
		\pause
		where $\varphi_t$ gives the population distribution at time $t$, the limits $L$, $U$ are the lower- and upper-limits of the structure variable $x$, and the kernel $k(y, x)$ determines how individuals of size $x$ contribute to those of size $y$ in the next time step.
		\pause
		\item IPMs generalize Leslie matrices by allowing for a continuous structure variable.
	\end{itemize}
\end{frame}

\begin{frame}
	\frametitle{Introduction}
	\begin{itemize}
		\item We will consider kernel functions of the form
		\[k(y, x) = s(x) g(y, x) + b(y)f(x),\]
		\pause
		where 
		\pause
		\begin{itemize}
			\item $s(x)$ is the survival function,
			\pause
			\item $g(y, x)$ is the growth subkernel,
			\pause
			\item $b(y)$ is the offspring distribution, 
			\pause
			\item and $f(x)$ the fecundity function.	
		\end{itemize}
	\end{itemize}
\end{frame}

\begin{frame}
	\frametitle{Introduction}
	\begin{itemize}
		\item When individuals are allowed to increase or decrease over the time step, the kernel function $k(y, x)$ of the IPM operator $A$ will be bounded.
		\pause
		\item For example, this is the case when the structure variable $x$ is stem diameter, or individual biomass.
		\pause
	\end{itemize}
\end{frame}

\begin{frame}
	\frametitle{Introduction}
		In this case, Ellner \& Rees, in appendices to \cite{Ellner2006}, proved the theorem:
	\pause
	\begin{theorem}
		Suppose $A$ is an IPM operator whose kernel $k(y, x)$ is positive and continuous on $[L, U]^2$. Then $\lambda = r(A)$ is an eigenvalue of $A$, and its eigenvector $\psi$ can be scaled to be positive. Additionally, $\lambda$ is the asymptotic growth rate of the population, and $\psi$ is the stable stage distribution, in the sense that for any nonzero initial population $\varphi_0$,
		\[\lim_{n \to \infty} \frac{A^n \varphi_0}{\lambda^n} = C \psi, \]
		where $C>0$.
	\end{theorem}
\end{frame}

\begin{frame}
	\frametitle{Introduction}
	\begin{itemize}
		\item However, ecologists have since used IPMs to model individuals that \emph{cannot} shrink from one time step to the next.
		\pause
		\item For example, Vindenes et. al in \cite{Vindenes2014} modeled fish, and used length as the structure variable $x$.
		\pause
		\item Fish have bony skeletons, and hence cannot shrink in length.
	\end{itemize}
\end{frame}

\begin{frame}
	\frametitle{Introduction}
	\begin{itemize}
		\item Ellner \& Rees's theorem implicitly assumed that individuals could shrink; if they cannot shrink the kernel will be unbounded.
		\pause
		\item All IPMs assume that $g(\cdot, x)$ is a probability distribution; that is:
		\[\int_L^U g(y, x) \, dy = 1, \quad \text{for all } x \in [L,U].\]
		\pause
		\item The assumption that individuals cannot shrink is that
		\[g(y, x) = 0, \text{whenever } y < x.\]
	\end{itemize}
\end{frame}

\begin{frame}
	\frametitle{Introduction}
\begin{center}
	\includegraphics[width=\linewidth]{"Images/Growth Kernel Plot"}
\end{center}
\end{frame}

\begin{frame}
	\frametitle{Introduction}
	\begin{itemize}
		\item With an IPM which models a population whose individuals cannot shrink, we have some questions to answer:
		\pause
		\begin{itemize}
			\item Is the operator $T$ still compact?
			\pause
			\item Is $\lambda = r(T)$ still an eigenvalue of $T$?
			\pause
			\item Are $\lambda$ and its eigenvector $\psi$ still the asymptotic growth rate and stable stage distribution, respectively, of the population?
		\end{itemize}
	\end{itemize}
\end{frame}

\begin{frame}
	\frametitle{Introduction}
	\begin{itemize}
		\item To be clear, our operators act on the space $L^1 = L^1(\Omega)$ of integrable functions on $\Omega:=[L,U]$. This is the natural space to work in, because the $L^1$-norm of a population distribution gives its total size.
		\pause
		\item Let $G:L^1 \to L^1$ denote the integral operator with kernel $g(y, x)$, the growth subkernel in the IPM:
		\pause
		\[(G\varphi)(y) := \int_L^U g(y, x) \, dx.\]
		\pause
		\item Let $S:L^1 \to L^1$ be multiplication by $s(x)$.
		\pause
		\item Let $F : L^1 \to \R$ be the fecundity functional $F\varphi : = \int_L^U f(x) \varphi(x) \, dx$.
		\pause
		\item Then the IPM operator $T$ can be written $T = GS + bF$, where $b = b(y)$ is the offspring distribution.
	\end{itemize}
\end{frame}

\begin{frame}
	\frametitle{$T$ is Not Compact}
		\begin{theorem}[Reichenbach, 2018]
			If $g(y, x)$ is the growth subkernel for an IPM which satisfies $g(y, x) = 0$ for $y < x$, then its associated integral operator $G$ is not compact.
		\end{theorem}
	\pause
	\begin{corollary}
		The IPM operator $T:= GS + bF$ is not compact.
	\end{corollary}
	\pause
	\begin{proof}[Proof Sketch of Corollary]
		An assumption of $s(x)$ is that $0 < s_0 \leq s(x)$ for all $x \in [L, U]$, so we can write
		\[G = \frac{T - bF}{s(x)}.\]
		Hence, $G$ and $T$ must be compact/non-compact together.
	\end{proof}
\end{frame}

\begin{frame}
	\frametitle{$T$ is Not Compact}
		\begin{proof}[Proof Sketch for Theorem]
			\begin{itemize}
			\item To show that $G$ is not compact, we can show something stronger: $G$ is not \emph{weakly} compact.
			\pause
			\item Let $\mathscr{U} \subset L^1$ be the unit ball. A theorem in Dunford \& Schwartz \cite{Dunford1958} says that the set $G(\mathscr{U})$ is weakly compact on $L^1(\Omega)$ iff
			\begin{equation}
				\lim_{\mu(E) \to 0} \int_E (G\varphi)(t) \, dt = 0 \label{weakcompact}
			\end{equation}
			uniformly for $\varphi \in \mathscr{U}$, where $\mu$ is the Lebesgue measure.
			\pause
			\item Put $\dd_n := \frac 1n (U-L)$, and define $E_n:= [U-\dd_n, U]$; then $\mu(E_n) \to 0$.
			\pause
			\item Define $h_n(t) : = \frac{1}{\dd_n} \chi_{E_n}(t)$; then $h_n \in \mathscr{U}$. The limit \eqref{weakcompact} is not uniform on the collection $\{h_n\}$. Therefore, $G$ is not weakly compact.
			\end{itemize}
		\end{proof}
\end{frame}

\begin{frame}
	\frametitle{$T$ is Not Compact}
	\begin{itemize}
		\item When Ellner \& Rees proved their theorem about the compact IPM operator $T$, they used a theorem that only required there to be an $N \in \N$ such that $T^k$ is compact for $k \geq N$.
		\pause
		\item However, the proof in the written dissertation shows that $T^k$ fails to be compact for all $k \in \N$ when $g(y, x) = 0$ for $y < x$.
		\pause
		\item Thus, we will require a wholly different method in order to prove a similar theorem.
	\end{itemize}
\end{frame}

\begin{frame}
	\frametitle{Helpful Definitions}
	\begin{definition}
		A closed convex set $K$ of the real Banach space $X$ is called a \emph{cone} if the following conditions hold:
		\begin{itemize}
			\item for any $x \in K$ and $a \geq 0$, the element $ax$ is in $K$,
			\pause
			\item for any pair $x$, $y \in K$, the element $x + y$ is in $K$, and
			\pause
			\item $K \cap -K = \{0\}$.
		\end{itemize}
	\end{definition}
	\pause
	For example, the set of nonnegative almost-everywhere functions in $L^1$ forms a cone.
	\pause
	\begin{definition}
		For a Banach space $X$, its \emph{Banach dual space} $X^*$ is the collection of continuous linear functionals on $X$.
	\end{definition}
	\pause
	For example, $L^\infty$ is the Banach dual of $L^1$.
\end{frame}

\begin{frame}
	\frametitle{Helpful Definitions}
	\begin{definition}
		Given a bounded operator $T:X \to X$, its \emph{spectrum} $\sigma(T) \subset \C$ is the set
		\[\sigma(T):=\{z \in \C \mid zI - T \text{ is not boundedly invertible}\}.\]
		\pause
		The \emph{spectral radius} $r(T)$ is given by
		\[r(T):= \sup \{ |z| \mid z \in \sigma(T)\}.\]
	\end{definition}
		\pause
	\begin{definition}
		Given an operator $T:X \to X$, its \emph{resolvent} $R(z, T)$ is defined as $(zI - T)^{-1}$, which is holomorphic in the \emph{resolvent set} $\rho(T) := \C \setminus \sigma(T)$.
	\end{definition}
\end{frame}

\begin{frame}
	\frametitle{Helpful Definitions}
	\begin{definition}
		Given a cone $K$ in a Banach space $X$, the \emph{dual cone} $K^* \subset X^*$ is the collection of all continuous linear functionals $x^*$ such that $\langle x, x^* \rangle \geq 0$ for all $x \in K$.
	\end{definition}
	\pause
	For example, if $K \subset L^1$ is the standard cone, then $K^* \subset L^\infty$ is the collection of functionals represented by nonnegative functions.
	\pause
	\begin{definition}
		An operator $T:X \to X$ is called \emph{positive} with respect to the cone $K \subset X$ if $T(K) \subset K$.
	\end{definition}
\end{frame}

\begin{frame}
	\frametitle{Helpful Definitions}
		\begin{definition}
			Given a linear operator $T:X \to X$, its \emph{Banach adjoint} $T^*:X^* \to X^*$ is the unique operator such that $\langle Tx, x^* \rangle = \langle x, T^*x^* \rangle$. 
		\end{definition}
	\pause
	For example, given an integral operator $T:L^1 \to L^1$ defined by
	\[(T\varphi)(y) = \int_L^U k(y, t) \varphi(t) \, dt,\]
	\pause
	its Banach adjoint is given by ``transposing" the kernel function:
	\[(T^*\varphi^*)(t) = \int_L^U k(y, t)\varphi^*(y) \, dy.\]
\end{frame}

\begin{frame}
	\frametitle{Helpful Definitions}
		\begin{definition}
			Given a cone $K$, an element $\varphi \in K$ is called \emph{quasi-interor} if $\langle \varphi, \varphi^* \rangle >0$ for all nonzero $\varphi^* \in K^*$
		\end{definition}
		\pause
		For example, the quasi-interior elements of $K \subset L^1$ are the positive almost-everywhere integrable functions.
		\pause
		\begin{definition}
			Given a cone $K$, an element $\varphi^* \in K^*$ is called \emph{strictly positive} if $\langle \varphi, \varphi^* \rangle > 0$ for all nonzero $\varphi \in K$.
		\end{definition}
		\pause
		For the same cone $K$ above, the strictly positive elements of $K^* \subset L^\infty$ are those represented by positive almost-everywhere, essentially bounded functions.
\end{frame}

\begin{frame}
	\frametitle{Main Theorem}
	\begin{theorem}[R., 2020]
		Suppose that $T:L^1 \to L^1$ is an IPM operator such that $g(y, x) = 0$ for $y < x$. Then under biologically reasonable assumptions, $T$ has the following properties (among others):
		\pause
		\begin{itemize}
			\item The spectral radius $\lambda = r(T)$ is a positive eigenvalue for $T$ and $T^*$. Moreover, the respective eigenvectors $\psi$ and $\psi^*$ span one-dimensional eigenspaces, $\psi$ is quasi-interior, $\psi^*$ represents a strictly positive linear functional, and both $\psi$, $\psi^*$ are the only eigenvectors of $T$, $T^*$ which can be scaled to be nonnegative almost-everywhere.
			\pause
			\item Suppose $\psi$ is scaled so that $||\psi||_1 = 1$, and $\psi^*$ is scaled so that $\langle \psi, \psi^* \rangle = 1$. Then for any nonzero $\psi_0 \in K$, we have
			\[\lim_{n \to \infty} \frac{T^n \varphi_0}{\lambda^n} = \langle \varphi_0, \psi^* \rangle \psi.\]			
		\end{itemize}
	\end{theorem}
\end{frame}

\begin{frame}
	\frametitle{Main Theorem}
	\begin{itemize}
		\item This theorem is in large part a consequence of a general theorem of Marek \cite{Marek1970}, though the tools he used were introduced by Sawashima \cite{Sawashima1964}.
		\pause
		\item To apply Marek's theorem, we need to show two things:
		\begin{enumerate}
			\pause
			\item $T$ is a \emph{nonsupporting} operator, and
			\pause
			\item $r(T)$ is a pole of the resolvent $R(z, T)$.
		\end{enumerate}
	\end{itemize}
\end{frame}

\begin{frame}
	\frametitle{$T$ is Nonsupporting}
		\begin{definition}
			Suppose $T:X \to X$ is a positive operator with respect to the cone $K$, and suppose that $\varphi \in K$, $\varphi^* \in K^*$ are both nonzero.
			\begin{itemize}
				\pause
				\item $T$ is called \emph{nonsupporting} if for every pair $\varphi$, $\varphi^*$ there exists a positive integer $p = p(\varphi, \varphi^*)$ such that $\langle T^n \varphi, \varphi^* \rangle > 0$ for every $n \geq p$.
				\pause
				\item $T$ is called \emph{strictly nonsupporting} if for every pair $\varphi$, $\varphi^*$ there is a positive integer $p = p(\varphi)$ such that $\langle T^n \varphi, \varphi^* \rangle > 0$ for $n \geq p$.
			\end{itemize}
		\end{definition}
	\pause
	Our non-compact IPM operator $T:L^1 \to L^1$ is in fact strictly non-supporting, and the integer $p$ actually doesn't depend on the choice of $\varphi \in L^1$ either.
\end{frame}

\begin{frame}
	\frametitle{T is Nonsupporting}
	\begin{itemize}
		\item These concepts are useful because they are applicable to a general cone $K$.
		\pause
		\item But in our particular case of $K \subset L^1$, an operator $T$ is strictly nonsupporting if $T^n \varphi$ is positive almost-everywhere for every nonzero $\varphi \in K$ and sufficiently large $n$.
		\pause
		\item An easy condition to guarantee this is to assume $b(y) > 0$ almost everywhere.
		\pause
		\item However, this is not a biologically realistic assumption. We instead imposed more conditions on the growth kernel $g(y, x)$, and which all IPMs satisfy (to our knowledge) in order to prove:
		\pause
		\begin{theorem}
			The IPM operator $T$ is strictly nonsupporting (hence, nonsupporting) under biologically reasonable assumptions. Additionally, the integer $p$ does not depend on the choice of $\varphi \in L^1$.
		\end{theorem}
	\end{itemize}
\end{frame}

\begin{frame}
	\frametitle{$r(T)$ is a Pole of the Resolvent}
	\begin{itemize}
		\item Recall that the \emph{resolvent} of an operator $T$ is defined to be $R(z, T) := (zI - T)^{-1}$, which is well-defined in the resolvent set $\rho(T) := \C \setminus \sigma(T)$.
		\pause
		\item To show that $r(T)$ is a pole of $R(z, T)$, we will apply the theorem
	\end{itemize}
	\pause
	\begin{theorem}[Clement, 1987 \cite{Clement1987}]
		Suppose that $z \in \sigma(T) \setminus \sigma_e(T)$, where $\sigma_e(T)$ denotes the \emph{essential spectrum}. Then $z$ is a pole of $R(z, T)$.
	\end{theorem}
	\begin{itemize}
		\item With this theorem, you can intuitively think of the essential spectrum as ``the points in the spectrum that are not poles".
	\end{itemize}
\end{frame}

\begin{frame}
	\frametitle{$r(T)$ is a Pole of the Resolvent}
	\begin{definition}
		We say an element $z \in \sigma(T)$ is in the \emph{essential spectrum}, denoted $\sigma_e(T)$, if one of the following conditions holds:
		\begin{enumerate}
			\pause
			\item The range of $(zI - T)$ is not closed,
			\pause
			\item $z$ is a limit point $\sigma(T)$, or
			\pause
			\item $\cup_{n=1}^\infty \ker(zI - T)^n$ is infinite-dimensional.
		\end{enumerate}
			\pause
			The \emph{essential spectral radius} is the value
			\[r_e(T) := \sup\{|z| \mid z \in \sigma_e(T)\}.\]
	\end{definition}
\end{frame}

\begin{frame}
	\frametitle{$r(T)$ is a Pole of the Resolvent}
	\begin{itemize}
		\item Hereafter, put $\lambda = r(T)$. To show the statement in the title of this slide, we need to show:
		\begin{enumerate}
			\pause
			\item $\lambda \in \sigma(T)$, and
			\pause
			\item $\lambda \not \in \sigma_e(T)$.
		\end{enumerate}
	\pause
	\item The first inclusion is straightforward:
	\end{itemize}
	\pause
	\begin{theorem}[Schaefer (1960), \cite{Schaefer1960}]
		Let $K \subset X$ be a normal cone. If $A:X \to X$ is a positive operator with respect to $K$, then $\lambda = r(A)$ is an element of $\sigma(A)$.
	\end{theorem}
	\pause
	\begin{itemize}
		\item The cone $K \subset L^1$ of nonnegative a.e. functions is normal, and the IPM operator $T:L^1 \to L^1$ is positive w.r.t. $K$, so $\lambda \in \sigma(T)$.
	\end{itemize}
\end{frame}

\begin{frame}
	\frametitle{$r(T)$ is a Pole of the Resolvent}
	\begin{itemize}
		\item To show that $\lambda \not \in \sigma_e(T)$, we will exhibit a real-valued $\mu \in \sigma(T)$ such that $\mu > r_e(T)$.
		\pause
		\item This implies that
		\[\lambda \geq \mu > r_e(T),\]
		\pause
		which in turn implies that $\lambda \not \in \sigma_e(T)$.
	\end{itemize}
\end{frame}

\begin{frame}
	\frametitle{$r(T)$ is a Pole of the Resolvent}
	\begin{itemize}
		\item The first step in this process is to compute $r_e(T)$.
		\pause
		\item There is a formula, due to Nussbaum \cite{Nussbaum1970} for $r_e(T)$:
		\[r_e(T) = \lim_{n \to \infty} \beta(T^n)^{1/n},\]
		\pause
		where $\beta(T^n) := \beta(T^n(\mathscr{U}))$, and $\beta$ is the \emph{ball measure of noncompactness} (or ball-MNC).
	\end{itemize}
\end{frame}

\begin{frame}
	\frametitle{$r(T)$ is a Pole of the Resolvent}
	\begin{definition}
		The \emph{ball-MNC}, also known as the \emph{Hausdorff-MNC}, of a subset $V$ of the vector space $X$ is given by
		\[\beta(V) := \inf\{r > 0 \mid V \text{ can be covered by finitely many balls of radius } r\}\]
	\end{definition}
	\pause
	\begin{itemize}
		\item for $V \subset L^p$, there is a ``nice" formula for $\beta(V)$:
			\[\beta(V) = \frac 12 \lim_{\dd \to 0} \sup_{\varphi \in V} \sup_{0 < \tau \leq \dd} ||\varphi - \varphi_\tau||_p,\]
			where $\varphi_\tau(t) := \varphi(t+\tau)$.
	\end{itemize}
\end{frame}

\begin{frame}
	\frametitle{$r(T)$ is a Pole of the Resolvent}
	\begin{itemize}
		\item Since the mapping $bF:L^1 \to L^1$ has a one-dimensional range (spanned by $b = b(y))$,  it's compact.
		\pause
		\item Hence, the ball-MNC $\beta$ ``doesn't see it", and
		\[\beta(T^k) = \beta((GS + bF)^k) = \beta((GS)^k).\] 
		\vspace*{-\baselineskip}
		\pause
		\item This simplifies the problem a lot, and allows us to compute
		\[r_e(T) = \lim_{n \to \infty} \beta(T^n)^{1/n} = \lim_{n \to \infty} \beta((GS)^n)^{1/n} = r_e((GS)^n)^{1/n}.\]
	\end{itemize}
\end{frame}

\begin{frame}
	\frametitle{$r(T)$ is a Pole of the Resolvent}
		\begin{theorem}[R., 2018]
			Put $s_1:= \sup\{s(x)\} = s(U)$. Then
			\begin{equation}
				r_e(GS) \leq r(GS) \leq s_1. \label{radii}
			\end{equation}
			If $g(y, x) = 0$ for $y < x$, then equalities hold in \eqref{radii}.
		\end{theorem}
	\pause
	\begin{proof}[Proof sketch]
		The first inequality follows from the fact that $\sigma_e(GS) \subset \sigma(GS)$, and the second follows from Gelfand's formula for the spectral radius:
		\[r(GS) = \lim_{n \to \infty} ||(GS)^n||^{1/n}.\] \pause Assuming that $g(y, x) = 0$ for $y < x$, one can use the formula for $\beta$, with a properly chosen subsequence of functions in $L^1$, to show that $s_1 \leq r_e(GS)$.		
	\end{proof}
\end{frame}

\begin{frame}
	\frametitle{$r(T)$ is a Pole of the Resolvent}
	\begin{theorem}
		Suppose $\mu \in \rho(GS)$, the resolvent set of $GS$, and define $\psi:=(\mu I - GS)^{-1} b$. If
		\[F \psi = F(\mu I - GS)^{-1} b = 1,\]
		\pause
		then $\psi$ is an eigenvector for $T$ with eigenvalue $\mu$.
		\pause
		Conversely, if $v$ is an eigenvector for $T$ with eigenvalue $\mu \in \rho(GS)$, then $v$ is in the span of $\psi$, and $F \psi = 1$.
	\end{theorem}
	\pause
	\begin{itemize}
	\item This characterizes what eigenvectors look like, and tells us exactly when $T$ has an eigenvalue $\mu$ (so long as $\mu \in \rho(GS)$).
	\pause
	\item Recall that we want some $\mu$ such that
	\[r(T) \geq \mu > r_e(T) = s_1.\]
	\end{itemize}
\end{frame}

\begin{frame}
	\frametitle{$r(T)$ is a Pole of the Resolvent}
	\begin{theorem}[R., 2019]
		Put $E:=(s_1, \infty)$, and define the function $P:E \to \R$ by
		\[P(t):= F(t I - GS)^{-1}b,\]
		where $GS$ satisfies biologically reasonable properties. Then:
		\pause
		\begin{enumerate}
			\item $P$ is continuous,
			\pause
			\item $P$ is strictly decreasing,
			\pause
			\item $\lim_{t \to \infty} P(t) = 0$,
			\pause
			\item and if in addition there is an $\ee > 0$ such that $s(x) \equiv s_1$ for $x \in [U-\ee, U]$, then
			\[\lim_{t \to s_1} P(t) = \infty.\]
		\end{enumerate}
	\end{theorem}
\end{frame}

\begin{frame}
	\frametitle{$r(T)$ is a Pole of the Resolvent}
	Here's what $P(t)$ might look like because of this theorem:
	\pause
\begin{center}
	\includegraphics[width=0.8\linewidth]{Images/P(t)_Plot}
\end{center}
\end{frame}

\begin{frame}
	\frametitle{$r(T)$ is a Pole of the Resolvent}
	\begin{itemize}
		\item This theorem guarantees the existence of a (unique) $\mu > s_1$ such that
		\[P(\mu) = F(\mu I - GS)^{-1}b = 1.\]
		\vspace*{-\baselineskip}
		\pause
		\item Hence, $\mu$ is an eigenvalue of $T$, so $\mu \in \sigma(T)$.
		\pause
		\item Additionally, for $\lambda = r(T)$,
		\[\lambda \geq \mu > s_1 = r_e(GS).\]
		\pause
		Thus, we can conclude that $\lambda \not \in \sigma_e(T)$. This is the last ingredient we needed to prove...
	\end{itemize}
\end{frame}

\begin{frame}
	\begin{theorem}[R., 2020]
	Suppose that $T:L^1 \to L^1$ is an IPM operator such that $g(y, x) = 0$ for $y < x$. Then under biologically reasonable assumptions, $T$ has the following properties (among others):
	\pause
	\begin{itemize}
		\item The spectral radius $\lambda = r(T)$ is a positive eigenvalue for $T$ and $T^*$. Moreover, the respective eigenvectors $\psi$ and $\psi^*$ span one-dimensional eigenspaces, $\psi$ is quasi-interior, $\psi^*$ represents a strictly positive linear functional, and both $\psi$, $\psi^*$ are the only eigenvectors of $T$, $T^*$ which can be scaled to be nonnegative almost-everywhere.
		\pause
		\item Suppose $\psi$ is scaled so that $||\psi||_1 = 1$, and $\psi^*$ is scaled so that $\langle \psi, \psi^* \rangle = 1$. Then for any nonzero $\psi_0 \in K$, we have
		\[\lim_{n \to \infty} \frac{T^n \varphi_0}{\lambda^n} = \langle \varphi_0, \psi^* \rangle \psi.\]			
	\end{itemize}
	\end{theorem}
\end{frame}


\begin{frame}
	\frametitle{Estimating the Spectral Radius}
	\begin{itemize}
		\item Before I state the result, define the functions
		\pause
		\begin{align*}
		Q(t) &:= -1 + F(t I - GS)^{-1} b = -1 + \sum_{k = 0}^\infty \frac{F((GS)^kb)}{t^{k+1}}, \\
		Q_n(t) &:= -1 + \sum_{k = 0}^n \frac{F((GS)^kb)}{t^{k+1}}, \\
		Q_{n, \dd}(t) &:= -1 + \sum_{k = 0}^n \frac{F((G_\dd S)^kb)}{t^{k+1}},
		\end{align*}
		\pause
		where $G_\dd$ is the integral operator with kernel equal to $g(y, x)$ on $[L, U-\dd] \times [L, U]$, and $0$ otherwise.
	\end{itemize}
\end{frame}

\begin{frame}
	\frametitle{Estimating the Spectral Radius}
	\begin{theorem}[R., 2020]
		For every $\ee>0$, there is an $N \in \N$ and a $\dd(N) >0$ such that for any $n \geq N$ and $\dd < \dd(N)$, we have
		\[|z_{n,\dd} - \lambda| < \ee, \]
		where $z_{n,\dd}$ is the unique zero of $Q_{n, \dd}$, and $\lambda$ is the unique zero of $Q$ (i.e., the spectral radius of $T$). 
	\end{theorem}
\end{frame}

\begin{frame}
	\frametitle{Questions or Comments?}
	\begin{enumerate}
		\pause
		\item Question?
		\begin{itemize}
			\pause
			\item Comment.
			\pause
			\item Comment.
		\end{itemize}
		\pause
		\item Question?
			\begin{itemize}
			\pause
			\item Comment.
			\pause
			\item Question?
				\begin{enumerate}
					\pause
					\item Comment.
					\pause
					\item Comment.
				\end{enumerate}
			\end{itemize}
	\end{enumerate}
\end{frame}

\begin{frame}[allowframebreaks]
	\frametitle{References}
		\printbibliography
\end{frame}

\begin{frame}
	\frametitle{Estimating the Spectral Radius}
	\begin{itemize}
		\item An important question is whether one can estimate $\lambda = r(T)$ on a computer when $T$ is not compact.
		\pause
		\item The standard methods of doing this when the operator is compact rely on approximating it uniformly with matrices.
		\pause
		\item But these methods do not work when $T$ is not compact.
		\pause
		\item I was not able to fully resolve this question, but I did prove that the spectral radii of certain compact operators approach the spectral radius of $T$.
	\end{itemize}
\end{frame}


%\begin{frame}
%	\frametitle{}
%	\begin{itemize}
%		\item 
%	\end{itemize}
%\end{frame}


\end{document}









